\documentclass[a4paper, 11pt]{article}

%\documentclass[10pt, a4paper, oneside
%headinclude,footinclude, % Extra spacing for the header and footer
%BCOR5mm,                 % Binding correction
%]{scrartcl}

\usepackage[nochapters,
beramono,        % Use the Bera Mono font for monospaced text (\texttt)
eulermath,
pdfspacing,      % Makes use of pdftex’ letter spacing capabilities via the microtype package
dottedtoc        % Dotted lines leading to the page numbers in the table of contents
]{classicthesis} % The layout is based on the Classic Thesis style

%\usepackage{arsclassica}            % заграничный шрифт
\usepackage{mathtext}                % русские буквы в формулах
\usepackage{cmap}                    % поиск в PDF
\usepackage[T2A]{fontenc}            % кодировка
\usepackage[utf8]{inputenc}          % кодировка исходного текста
\usepackage[english, russian]{babel} % локализация и переносы
\renewcommand{\rmdefault}{ftm}       % Times New Roman

\usepackage{amsmath, amsfonts, amssymb, amsthm}
\usepackage{verbatim}                % печать как есть
\usepackage{latexsym}
\usepackage{pscyr}                   % улучшенные русские шрифты
\usepackage{indentfirst}             % красная строка
\usepackage{icomma}                  % "Умная" запятая
\usepackage{lastpage}                % узнать сколько страниц в документе
\usepackage{misccorr}
\usepackage{enumitem}                % Required for manipulating the whitespace between and within lists

%% Номера формул
%\mathtoolsset{showonlyrefs=true} % Показывать номера только у тех формул, на которые есть \eqref{} в тексте.
%\usepackage{leqno}               % Нумерация формул слева

%% Перенос знаков в формулах (по Львовскому)
\newcommand*{\hm}[1]{#1\nobreak\discretionary{}
    {\hbox{$\mathsurround=0pt #1$}}{}}

%%% Изображения
\usepackage{graphicx}         % для вставки рисунков
	\graphicspath{{Figures/}} % папки с картинками
	\setlength\fboxsep{3pt}   % отступ рамки \fbox{} от рисунка
	\setlength\fboxrule{1pt}  % толщина линий рамки \fbox{}
\usepackage{wrapfig}          % обтекание рисунков текстом
%\renewcommand{\rmdefault}{ftm}
\usepackage{subfigure}
%\usepackage{subfig}
\usepackage{varioref}         % More descriptive referencing


%%% Работа с таблицами
\usepackage{array,tabularx,tabulary,booktabs} % дополнительная работа с таблицами
\usepackage{longtable}                        % длинные таблицы
\usepackage{multirow}                         % слияние строк в таблице
\usepackage{booktabs}                         % Таблицы как в книгах

%%% Работа с графикой
\usepackage{tikz}
\usepackage{pgfplots}
\usepackage{pgfplotstable}
\usepackage[all]{xy}

%%% Подписи под изображениями и таблицами
\usepackage[tableposition=top]{caption}
\usepackage{subcaption}
    \DeclareCaptionLabelFormat{gostfigure}{Рисунок #2}
    \DeclareCaptionLabelFormat{gosttable}{Таблица #2}
    \DeclareCaptionLabelSeparator{gost}{~---~}
    \captionsetup{labelsep=gost}
    \captionsetup[figure]{labelformat=gostfigure}
    \captionsetup[table]{labelformat=gosttable}
	\renewcommand{\thesubfigure}{\asbuk{subfigure}}
% Либо так:
\usepackage{caption}
\DeclareCaptionFormat{GOSTtable}{#2#1\\#3}
\DeclareCaptionLabelSeparator{fill}{\hfill}
\DeclareCaptionLabelFormat{fullparents}{\bothIfFirst{#1}{~}#2}
\captionsetup[table]{
    format=GOSTtable,
    font={footnotesize},
    labelformat=fullparents,
    labelsep=fill,
    labelfont=it,
    textfont=bf,
    justification=centering,
    singlelinecheck=false
 	}


%%% Страница
\usepackage{lscape}	  % для включения альбомных страниц
\usepackage{extsizes} % возможность сделать 14-й шрифт
\usepackage{geometry} % простой способ задавать поля
	\geometry{top=25mm}
	\geometry{bottom=35mm}
	\geometry{left=20mm}
	\geometry{right=20mm}
\usepackage{multicol}  % несколько колонок


%%% Интерлиньяж
\usepackage{setspace}
    \onehalfspacing % интерлиньяж 1.5
    %\doublespacing  % интерлиньяж 2
    %\singlespacing  % интерлиньяж 1
%\linespread{1.3}     % полуторный интервал в Word
%\renewcommand{\baselinestretch}{1.65} % либо так


%%% Теоремы
\theoremstyle{plain}
    \newtheorem{theorem}{Теорема}
    \newtheorem{proposition}[theorem]{Утверждение}
\theoremstyle{definition}
    \newtheorem{definition}{Определение}[section]
    \newtheorem{corollary}{Следствие}[theorem]
    \newtheorem{problem}{Задача}[section]
\theoremstyle{remark} % "Примечание"
    \newtheorem*{nonum}{Решение}


%%% Программирование
\usepackage{etoolbox} % логические операторы


%%% Общее форматирование и оформление
\usepackage[singlelinecheck=off,center]{caption}	% Многострочные подписи
\usepackage{soul}      % печать в разрядку, подчёркивание, перечёркивание текста и многое друго
\usepackage{decimal}   % изменить поведение разделителя между целой и дробной частями числа в зависимости от режима
\usepackage{microtype} % висячая пунктуация
\usepackage{iitem}     % макросы \iitem, \iiitem и \ivtem, формирующий многоуровневый перечень
\usepackage{makecell}
\frenchspacing
\usepackage{ulem}      % подчеркивания
\usepackage{tocloft}   % для оглавления
\renewcommand{\cftsecleader}{\cftdotfill{\cftdotsep}} % точки вместо пробелов в оглавлении
%\renewcommand{\familydefault}{\sfdefault} % Начертание шрифта

% Позволяет менять уровень отображения разделов в оглавлении там, где надо
\usepackage{tocvsec2}
\usepackage[usenames, dvinames]{color} % цвета
\usepackage{colortbl}

%%% Колонтитулы
\usepackage{fancyhdr}
 	\pagestyle{fancy}
 	\renewcommand{\headrulewidth}{0pt}  % толщина линейки, отчеркивающей верхний колонтитул
 	\lfoot{Нижний левый}
 	\rfoot{Нижний правый}
 	\rhead{Верхний правый}
 	\chead{Верхний в центре}
 	\lhead{Верхний левый}
	\cfoot{Нижний в центре}            % По умолчанию здесь номер страницы

%%% Гиперссылки %%%
\usepackage{csquotes}       % Еще инструменты для ссылок
\usepackage{hyperref}
\usepackage[usenames,dvipsnames,svgnames,table,rgb]{xcolor}
\hypersetup{				% гиперссылки
	%draft, 				% Uncomment to remove all links (useful for printing in black and white)
    unicode=true,           % русские буквы в раздела PDF
    pdftitle={Заголовок},   % заголовок
    pdfauthor={\textcopyright},      % автор
    pdfsubject={Тема},      % тема
    pdfcreator={Создатель}, % создатель
    pdfproducer={Производитель}, % производитель
    pdfkeywords={keyword1} {key2} {key3}, % ключевые слова
    colorlinks=true,       	% false: ссылки в рамках; true: цветные ссылки
    breaklinks=true,
    bookmarks=true,
    bookmarksnumbered,
	urlcolor=webbrown,
	linkcolor=RoyalBlue,
    citecolor=webgreen,     % на библиографию
    filecolor=magenta,      % на файлы
    urlcolor=cyan           % на URL
}
%\usepackage[plainpages=false,pdfpagelabels=false]{hyperref}
%\definecolor{linkcolor}{rgb}{0.9,0,0}
%\definecolor{urlcolor}{rgb}{0,0,1}
%\hypersetup{
%    colorlinks, linkcolor={linkcolor},
%    urlcolor={urlcolor}
%}

\hyphenation{Fortran hy-phen-ation} % Specify custom hyphenation points in words with dashes where you would like hyphenation to occur, or alternatively, don't put any dashes in a word to stop hyphenation altogether

\title{\normalfont\spacedallcaps{Название статьи}}
\author{\spacedlowsmallcaps{John Smith* \& Шведов Олег\textsuperscript{1}}}
\date{}


%%% Библиография
\usepackage{thebibliorgaphy}
\usepackage[square,numbers,sort&compress]{natbib}
\renewcommand{\bibnumfmt}[1]
{#1.\hfill} % нумерация источников в самом списке — через точку
\renewcommand{\bibsection}
{\likechapter{Список литературы}} % заголовок специального раздела
\setlength{\bibsep}{0pt}

\usepackage{cite} % красивые ссылки на литературу
\makeatletter
\bibliographystyle{utf8gost705u}  % оформляем библиографию в соответствии с ГОСТ 7.0.5
\renewcommand{\@biblabel}[1]{#1.} % заменяем библиографию с квадратных скобок на точку
\makeatother

%%% BibLaTeX
\usepackage[backend=biber,bibencoding=utf8,sorting=ynt,maxcitenames=2,style=authoryear]{biblatex}
\addbibresource{bib1.bib}
\addbibresource{bib2.bib}

%%% Выравнивание и переносы
\sloppy					% Избавляемся от переполнений
\clubpenalty=10000		% Запрещаем разрыв страницы после первой строки абзаца
\widowpenalty=10000		% Запрещаем разрыв страницы после последней строки абзаца
\frenchspacing
\tolerance = 300
\righthyphenmin = 2 % перенос последних двух букв допустим
\doublehyphendemerits = 35000 % нежелательность множества переносов подряд


%%% Создание новых окружений для списков с уменьшенными отступами
\newenvironment{itemize*}%
  {\begin{itemize}%
    \setlength{\itemsep}{1pt}%
    \setlength{\parskip}{1pt}}%
  {\end{itemize}}
\newenvironment{enumerate*}%
  {\begin{enumerate}%
    \setlength{\itemsep}{1pt}%
    \setlength{\parskip}{1pt}}%
  {\end{enumerate}}


%%% Новые команды для постановки ссылки на сноску и для ссылки на сноску
\newcommand{\footnoteremember}[2]{\footnote{#2} \newcounter{#1} \setcounter{#1}{\value{footnote}}}
\newcommand{\footnoterecall}[1]{\footnotemark[\value{#1}]}


%%% Переименовать оглавление
\renewcommand\contentsname{Содержание}


%%% Начинать новую главу с той же строки
\makeatletter
\renewcommand\chapter{\par%
\thispagestyle{plain}% \global\@topnum\z@
\@afterindentfalse \secdef\@chapter\@schapter}
\makeatother


%%% Поставить точку в названии главы или раздела
\renewcommand{\thechapter}{\arabic{chapter}.}
\renewcommand{\thesubsection}{\arabic{section}.\arabic{subsection}.}


%%% Переопределения мат. символов в традиционные русские
\renewcommand{\epsilon}{\ensuremath{\varepsilon}}
\renewcommand{\phi}{\ensuremath{\varphi}}
\renewcommand{\kappa}{\ensuremath{\varkappa}}
\renewcommand{\le}{\ensuremath{\leqslant}}
\renewcommand{\leq}{\ensuremath{\leqslant}}
\renewcommand{\ge}{\ensuremath{\geqslant}}
\renewcommand{\geq}{\ensuremath{\geqslant}}
\renewcommand{\emptyset}{\varnothing}


%%% XeLaTeX
\usepackage{polyglossia}
\setdefaultlanguage[babelshorthands=true]{russian}
\setotherlanguage{english}

\usepackage{fontspec} % подготавливает загрузку шрифтов Open Type, True Type
\defaultfontfeatures{Ligatures={TeX},Renderer=Basic}  % свойства шрифтов по умолчанию
\setmainfont[Ligatures={TeX,Historic}]{Times New Roman} % основной шрифт документа
\setsansfont{Comic Sans MS} % шрифт без засечек
\setmonofont{Courier New}

\newfontfamily\cyrillicfontrm{Times New Roman} % для пакета polyglossia
\newfontfamily\cyrillicfontsf{Comic Sans MS}
\newfontfamily\cyrillicfonttt{Courier New}
\newfontfamily\cyrillicfont{Times New Roman}


%%% Listings
\usepackage{listings}

\definecolor{string}{HTML}{B40000}  % цвет строк в коде
\definecolor{comment}{HTML}{008000} % цвет комментариев в коде
\definecolor{keyword}{HTML}{1A00FF} % цвет ключевых слов в коде
\definecolor{morecomment}{HTML}{8000FF} % цвет include и других элементов в коде
\definecolor{сaptiontext}{HTML}{FFFFFF} % цвет текста заголовка в коде
\definecolor{сaptionbk}{HTML}{999999}   % цвет фона заголовка в коде
\definecolor{bk}{HTML}{FFFFFF}       % цвет фона в коде
\definecolor{frame}{HTML}{999999}    % цвет рамки в коде
\definecolor{brackets}{HTML}{B40000} % цвет скобок в коде

\lstset{
    language = C++, % Язык кода по умолчанию
    keepspaces=true,
    morekeywords = {*,...}, % если хотите добавить ключевые слова, то добавляйте
        % Цвета
    keywordstyle = \color{keywordcolor}\ttfamily\bfseries,
    stringstyle = \color{stringcolor}\ttfamily,
    commentstyle = \color{commentcolor}\ttfamily\itshape,
    morecomment=[l][\color{morecommentcolor}]{\#},
        % Настройки отображения
    breaklines = true, % Перенос длинных строк
    basicstyle = \ttfamily\footnotesize, % Шрифт для отображения кода
    backgroundcolor = \color{bk}, % Цвет фона кода
    frame = single, xleftmargin = \fboxsep, xrightmargin = -\fboxsep, % Рамка, подогнанная к заголовку
    rulecolor = \color{framecolor}, % Цвет рамки
    tabsize = 4, % Размер табуляции в пробелах
    numbers = left, % Слева отображаются номера строк
    stepnumber = 1, % Каждую строку нумеровать
    numbersep = 5pt, % Отступ от кода
    numberstyle = \small\color{black}, % Стиль написания номеров строк
        % Для отображения русского языка
    extendedchars = true,
    literate =
      {Ö}{{\"O}}1
      {Ä}{{\"A}}1
      {Ü}{{\"U}}1
      {ß}{{\ss}}1
      {ü}{{\"u}}1
      {ä}{{\"a}}1
      {ö}{{\"o}}1
      {~}{{\textasciitilde}}1
      {а}{{\selectfont\char224}}1
      {б}{{\selectfont\char225}}1
      {в}{{\selectfont\char226}}1
      {г}{{\selectfont\char227}}1
      {д}{{\selectfont\char228}}1
      {е}{{\selectfont\char229}}1
      {ё}{{\"e}}1
      {ж}{{\selectfont\char230}}1
      {з}{{\selectfont\char231}}1
      {и}{{\selectfont\char232}}1
      {й}{{\selectfont\char233}}1
      {к}{{\selectfont\char234}}1
      {л}{{\selectfont\char235}}1
      {м}{{\selectfont\char236}}1
      {н}{{\selectfont\char237}}1
      {о}{{\selectfont\char238}}1
      {п}{{\selectfont\char239}}1
      {р}{{\selectfont\char240}}1
      {с}{{\selectfont\char241}}1
      {т}{{\selectfont\char242}}1
      {у}{{\selectfont\char243}}1
      {ф}{{\selectfont\char244}}1
      {х}{{\selectfont\char245}}1
      {ц}{{\selectfont\char246}}1
      {ч}{{\selectfont\char247}}1
      {ш}{{\selectfont\char248}}1
      {щ}{{\selectfont\char249}}1
      {ъ}{{\selectfont\char250}}1
      {ы}{{\selectfont\char251}}1
      {ь}{{\selectfont\char252}}1
      {э}{{\selectfont\char253}}1
      {ю}{{\selectfont\char254}}1
      {я}{{\selectfont\char255}}1
      {А}{{\selectfont\char192}}1
      {Б}{{\selectfont\char193}}1
      {В}{{\selectfont\char194}}1
      {Г}{{\selectfont\char195}}1
      {Д}{{\selectfont\char196}}1
      {Е}{{\selectfont\char197}}1
      {Ё}{{\"E}}1
      {Ж}{{\selectfont\char198}}1
      {З}{{\selectfont\char199}}1
      {И}{{\selectfont\char200}}1
      {Й}{{\selectfont\char201}}1
      {К}{{\selectfont\char202}}1
      {Л}{{\selectfont\char203}}1
      {М}{{\selectfont\char204}}1
      {Н}{{\selectfont\char205}}1
      {О}{{\selectfont\char206}}1
      {П}{{\selectfont\char207}}1
      {Р}{{\selectfont\char208}}1
      {С}{{\selectfont\char209}}1
      {Т}{{\selectfont\char210}}1
      {У}{{\selectfont\char211}}1
      {Ф}{{\selectfont\char212}}1
      {Х}{{\selectfont\char213}}1
      {Ц}{{\selectfont\char214}}1
      {Ч}{{\selectfont\char215}}1
      {Ш}{{\selectfont\char216}}1
      {Щ}{{\selectfont\char217}}1
      {Ъ}{{\selectfont\char218}}1
      {Ы}{{\selectfont\char219}}1
      {Ь}{{\selectfont\char220}}1
      {Э}{{\selectfont\char221}}1
      {Ю}{{\selectfont\char222}}1
      {Я}{{\selectfont\char223}}1
      {і}{{\selectfont\char105}}1
      {ї}{{\selectfont\char168}}1
      {є}{{\selectfont\char185}}1
      {ґ}{{\selectfont\char160}}1
      {І}{{\selectfont\char73}}1
      {Ї}{{\selectfont\char136}}1
      {Є}{{\selectfont\char153}}1
      {Ґ}{{\selectfont\char128}}1
      {\{}{{{\color{bracketscolor}\{}}}1
      {\}}{{{\color{bracketscolor}\}}}}1
    }
% Для настройки заголовка кода
%\usepackage{caption}
%    \DeclareCaptionFont{white}{\color{сaptiontext}}
%    \DeclareCaptionFormat{listing}{\parbox{\linewidth}{\colorbox{сaptionbk}{\parbox{\linewidth}{#1#2#3}}    \vskip-4pt}}
%    \captionsetup[lstlisting]{format=listing,labelfont=white,textfont=white}
\renewcommand{\lstlistingname}{Код}