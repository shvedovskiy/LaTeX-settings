\documentclass[a4paper, 10pt]{article}
%\documentclass[10pt, a4paper, oneside
%headinclude,footinclude,  % extra spacing for the header and footer
%BCOR5mm,                  % binding correction
%]{scrartcl}


%\usepackage{arsclassica}            % english font
\usepackage{mathtext}                % russian letters in formulas
\usepackage{cmap}                    % PDF search
\usepackage[T2A]{fontenc}
\usepackage[utf8]{inputenc}          % source text encoding
\usepackage[english, russian]{babel} % localization and hyphenation
\renewcommand{\rmdefault}{ftm}       % Times New Roman font (I prefer XeLaTeX for TNR font)


%%% Formatting $$$
\usepackage{verbatim}      % raw typing
\usepackage{pscyr}         % nice russian font
\usepackage{indentfirst}   % paragraph indent in Russian texts
\usepackage{lastpage}
\usepackage{misccorr}	   % an addition to babel's package
\usepackage{enumitem}      % for manipulating the whitespace between and within lists
\usepackage{etoolbox}      % logical operators
\usepackage{soul}          % detente, underlining, crossing etc.
\usepackage{microtype}     % висячая пунктуация (must use this dude)
\usepackage{iitem}         % macros for multilevel item lists (\iitem, \iiitem, \ivtem)
\usepackage{ulem}      	   % underline
\frenchspacing             % traditional Russian whitespaces
% New environment with reduced indentation:
\newenvironment{itemize*}%
  {\begin{itemize}%
    \setlength{\itemsep}{1pt}%
    \setlength{\parskip}{1pt}}%
  {\end{itemize}}
\newenvironment{enumerate*}%
  {\begin{enumerate}%
    \setlength{\itemsep}{1pt}%
    \setlength{\parskip}{1pt}}%
  {\end{enumerate}}


%%% Some math %%%
\usepackage{amsmath, amsfonts, amssymb, amsthm} % AMS-math packages
\usepackage{latexsym}							% extra math symbols
\usepackage{icomma}                             % smart comma
\usepackage{decimal}   							% changing of comma behavior depending on mode
%\mathtoolsset{showonlyrefs=true}               % show numbers only those formulas, which have \eqref{}
%\usepackage{leqno}                             % formulas numbers at left side
\newcommand*{\hm}[1]{#1\nobreak\discretionary{}
    {\hbox{$\mathsurround=0pt #1$}}{}}          % hyphenation for formulas
% Redefenition to traditional Russian math symbols:
\renewcommand{\epsilon}{\ensuremath{\varepsilon}}
\renewcommand{\phi}{\ensuremath{\varphi}}
\renewcommand{\kappa}{\ensuremath{\varkappa}}
\renewcommand{\le}{\ensuremath{\leqslant}}
\renewcommand{\leq}{\ensuremath{\leqslant}}
\renewcommand{\ge}{\ensuremath{\geqslant}}
\renewcommand{\geq}{\ensuremath{\geqslant}}
\renewcommand{\emptyset}{\varnothing}

    
%%% Images %%%
\usepackage{graphicx}         % main graphics package
	\graphicspath{{images/}}  % containing folders 
	\setlength\fboxsep{3pt}   % indent frame \fbox{} from the image
	\setlength\fboxrule{1pt}  % frame \fbox {}lines thickness 
\usepackage{wrapfig}          % text wrapping
%\renewcommand{\rmdefault}{ftm}
\usepackage{subfigure}
\usepackage{varioref}         % more descriptive referencing


%%% Tables %%%
\usepackage{array,tabularx,tabulary}  % extra table features
\usepackage{longtable}                % multipage tables
\usepackage{multirow}                 % merge rows in tables
\usepackage{multicol}				  % merge columns in tables
\usepackage{makecell}                 % com­mon lay­outs for tab­u­lar col­umn heads in whole doc­u­ments
%\usepackage{booktabs}                % beatiful professional tables


%%% Graphics %%%
\usepackage{tikz}
\usepackage{pgfplots}
\usepackage{pgfplotstable}
\usepackage[all]{xy}
\usepackage[usenames, dvinames]{color}
\usepackage{colortbl}


%%% Captions under the images and tables %%%
\usepackage[singlelinecheck=off,center]{caption}	
\usepackage{subcaption}
    \DeclareCaptionLabelFormat{gostfigure}{Рисунок #2}
    \DeclareCaptionLabelFormat{gosttable}{Таблица #2}
    \DeclareCaptionLabelSeparator{gost}{~---~}
    \captionsetup{labelsep=gost}
    \captionsetup[figure]{labelformat=gostfigure}
    \captionsetup[table]{labelformat=gosttable}
	\renewcommand{\thesubfigure}{\asbuk{subfigure}}
% Or this way:
%\usepackage{caption}
%\DeclareCaptionFormat{GOSTtable}{#2#1\\#3}
%\DeclareCaptionLabelSeparator{fill}{\hfill}
%\DeclareCaptionLabelFormat{fullparents}{\bothIfFirst{#1}{~}#2}
%\captionsetup[table]{
%    format=GOSTtable,
%    font={footnotesize},
%    labelformat=fullparents,
%    labelsep=fill,
%    labelfont=it,
%    textfont=bf,
%    justification=centering,
%    singlelinecheck=false}


%%% Page options %%%
\usepackage{lscape}	        % landscape paegs
\usepackage{extsizes}       % large fonts (14)
\usepackage{geometry}       % margins
	\geometry{top=25mm}
	\geometry{bottom=35mm}
	\geometry{left=20mm}
	\geometry{right=20mm}
\usepackage{multicol}{2}    % several text columns  
\usepackage{setspace}	    % line spacing	
    \onehalfspacing 
    %\doublespacing  
    %\singlespacing
% Other methods:
%\linespread{1.3}
%\renewcommand{\baselinestretch}{1.65}

\sloppy					% getting rid of the overflow
\clubpenalty=10000		% prevent a page break after the first line of the paragraph 
\widowpenalty=10000		% prevent a page break after the last line of the paragraph
\tolerance = 300
\righthyphenmin = 2 
\doublehyphendemerits = 35000

% Новые команды для постановки ссылки на сноску и для ссылки на сноску:
\newcommand{\footnoteremember}[2]{\footnote{#2} \newcounter{#1} \setcounter{#1}{\value{footnote}}}
\newcommand{\footnoterecall}[1]{\footnotemark[\value{#1}]}

% Начинать новую главу с той же строки
\makeatletter
\renewcommand\chapter{\par%
\thispagestyle{plain}% \global\@topnum\z@
\@afterindentfalse \secdef\@chapter\@schapter}
\makeatother

% Поставить точку в названии главы или раздела
\renewcommand{\thechapter}{\arabic{chapter}.}
\renewcommand{\thesubsection}{\arabic{section}.\arabic{subsection}.}


%%% Theorems %%%
\theoremstyle{plain}
    \newtheorem{theorem}{Теорема}
    \newtheorem{proposition}[theorem]{Утверждение}
\theoremstyle{definition}
    \newtheorem{definition}{Определение}[section]
    \newtheorem{corollary}{Следствие}[theorem]
    \newtheorem{problem}{Задача}[section]
\theoremstyle{remark}
    \newtheorem*{nonum}{Решение}


%%% Table of contents %%%
\renewcommand\contentsname{Содержание}                    % rename toc section
\usepackage{tocvsec2}                                     % chage depth of toc displaying
	\settocdepth{4}
\usepackage{tocloft}   
	\renewcommand{\cftsecleader}{\cftdotfill{\cftdotsep}} % dots between text and page number


%%% Running titles %%%
\usepackage{fancyhdr}
 	\pagestyle{fancy}
 	\renewcommand{\headrulewidth}{0pt}  % header line thickness
 	%\lfoot{Нижний левый}
 	%\rfoot{Нижний правый}
 	%\rhead{Верхний правый}
 	%\chead{Верхний в центре}
 	%\lhead{Верхний левый}
	%\cfoot{Нижний в центре}             % page number by default


%%% Hyperrefs %%%
\usepackage{csquotes}       % reference tools
\usepackage[plainpages=false,pdfpagelabels=false]{hyperref}
\usepackage[usenames,dvipsnames,svgnames,table,rgb]{xcolor}
	\definecolor{linkcolor}{rgb}{0.9,0,0}
	\definecolor{urlcolor}{rgb}{0,0,1}
\hypersetup{				% гиперссылки
	%draft, 				% uncomment to remove all links (useful for printing in black and white)
    unicode=true,           
    pdftitle={Заголовок},   
    pdfauthor={\textcopyright},     
    pdfsubject={Тема},      
    pdfcreator={Создатель}, 
    pdfproducer={Производитель},
    pdfkeywords={keyword1} {keyword2}, 
    colorlinks=true,       	% false: links in boxes; true: colored links
    breaklinks=true,
    bookmarks=true,
    bookmarksnumbered,
	urlcolor=webbrown,
	linkcolor={linkcolor},
    citecolor=black,     
    filecolor=magenta,      
    urlcolor={urlcolor}}


%%% Bibliography %%%
\usepackage{thebibliorgaphy}
\usepackage{cite}                 % beautiful cites
\usepackage[square,numbers,sort&compress]{natbib}
\renewcommand{\bibsection}{\likechapter{Список литературы}} % rename bibliography section
	\setlength{\bibsep}{0pt}
\makeatletter
\bibliographystyle{utf8gost705u}  % bibliography GOST
\renewcommand{\@biblabel}[1]{#1.} % numbering through the dot
\makeatother

%%% BibLaTeX %%%
\usepackage[backend=biber,bibencoding=utf8,sorting=ynt,maxcitenames=2,style=authoryear]{biblatex}
\addbibresource{bib1.bib}


%%% XeLaTeX
\usepackage{polyglossia}
\setdefaultlanguage[babelshorthands=true]{russian}
\setotherlanguage{english}

\usepackage{fontspec}                                   % loads Open Type, True Type fonts
\defaultfontfeatures{Ligatures={TeX},Renderer=Basic}    % default font references
\setmainfont[Ligatures={TeX,Historic}]{Times New Roman} % main document font
\setsansfont{Comic Sans MS}
\setmonofont{Courier New}
	\newfontfamily\cyrillicfontrm{Times New Roman}      % for polyglossia package
	\newfontfamily\cyrillicfontsf{Comic Sans MS}
	\newfontfamily\cyrillicfonttt{Courier New}
	\newfontfamily\cyrillicfont{Times New Roman}


%%% Listings
\usepackage{listings}
	\definecolor{string}{HTML}{B40000}      % code lines color
	\definecolor{comment}{HTML}{008000}     % comments color
	\definecolor{keyword}{HTML}{1A00FF}     % keywords color
	\definecolor{morecomment}{HTML}{8000FF} % include and other elements color
	\definecolor{сaptiontext}{HTML}{FFFFFF} % headline color
	\definecolor{сaptionbk}{HTML}{999999}   % headline's background color
	\definecolor{bk}{HTML}{FFFFFF}          % background color
	\definecolor{frame}{HTML}{999999}       % frame color
	\definecolor{brackets}{HTML}{B40000}    % brackets color

\lstset{
    language = Python,
    keepspaces=true,
    morekeywords = {*,...}, 
    keywordstyle = \color{keywordcolor}\ttfamily\bfseries,
    stringstyle = \color{stringcolor}\ttfamily,
    commentstyle = \color{commentcolor}\ttfamily\itshape,
    morecomment=[l][\color{morecommentcolor}]{\#},
    % displaying
    breaklines = true, % long lines
    basicstyle = \ttfamily\footnotesize, % code font
    backgroundcolor = \color{bk},
    frame = single, xleftmargin = \fboxsep, xrightmargin = -\fboxsep, % headline frame
    rulecolor = \color{framecolor}, % frame color
    tabsize = 4, 
    numbers = left, % numbers of lines
    stepnumber = 1, % number every line
    numbersep = 5pt,
    numberstyle = \small\color{black}, 
    % for cyrillic symbols
    extendedchars = true,
    literate =
      {Ö}{{\"O}}1
      {Ä}{{\"A}}1
      {Ü}{{\"U}}1
      {ß}{{\ss}}1
      {ü}{{\"u}}1
      {ä}{{\"a}}1
      {ö}{{\"o}}1
      {~}{{\textasciitilde}}1
      {а}{{\selectfont\char224}}1
      {б}{{\selectfont\char225}}1
      {в}{{\selectfont\char226}}1
      {г}{{\selectfont\char227}}1
      {д}{{\selectfont\char228}}1
      {е}{{\selectfont\char229}}1
      {ё}{{\"e}}1
      {ж}{{\selectfont\char230}}1
      {з}{{\selectfont\char231}}1
      {и}{{\selectfont\char232}}1
      {й}{{\selectfont\char233}}1
      {к}{{\selectfont\char234}}1
      {л}{{\selectfont\char235}}1
      {м}{{\selectfont\char236}}1
      {н}{{\selectfont\char237}}1
      {о}{{\selectfont\char238}}1
      {п}{{\selectfont\char239}}1
      {р}{{\selectfont\char240}}1
      {с}{{\selectfont\char241}}1
      {т}{{\selectfont\char242}}1
      {у}{{\selectfont\char243}}1
      {ф}{{\selectfont\char244}}1
      {х}{{\selectfont\char245}}1
      {ц}{{\selectfont\char246}}1
      {ч}{{\selectfont\char247}}1
      {ш}{{\selectfont\char248}}1
      {щ}{{\selectfont\char249}}1
      {ъ}{{\selectfont\char250}}1
      {ы}{{\selectfont\char251}}1
      {ь}{{\selectfont\char252}}1
      {э}{{\selectfont\char253}}1
      {ю}{{\selectfont\char254}}1
      {я}{{\selectfont\char255}}1
      {А}{{\selectfont\char192}}1
      {Б}{{\selectfont\char193}}1
      {В}{{\selectfont\char194}}1
      {Г}{{\selectfont\char195}}1
      {Д}{{\selectfont\char196}}1
      {Е}{{\selectfont\char197}}1
      {Ё}{{\"E}}1
      {Ж}{{\selectfont\char198}}1
      {З}{{\selectfont\char199}}1
      {И}{{\selectfont\char200}}1
      {Й}{{\selectfont\char201}}1
      {К}{{\selectfont\char202}}1
      {Л}{{\selectfont\char203}}1
      {М}{{\selectfont\char204}}1
      {Н}{{\selectfont\char205}}1
      {О}{{\selectfont\char206}}1
      {П}{{\selectfont\char207}}1
      {Р}{{\selectfont\char208}}1
      {С}{{\selectfont\char209}}1
      {Т}{{\selectfont\char210}}1
      {У}{{\selectfont\char211}}1
      {Ф}{{\selectfont\char212}}1
      {Х}{{\selectfont\char213}}1
      {Ц}{{\selectfont\char214}}1
      {Ч}{{\selectfont\char215}}1
      {Ш}{{\selectfont\char216}}1
      {Щ}{{\selectfont\char217}}1
      {Ъ}{{\selectfont\char218}}1
      {Ы}{{\selectfont\char219}}1
      {Ь}{{\selectfont\char220}}1
      {Э}{{\selectfont\char221}}1
      {Ю}{{\selectfont\char222}}1
      {Я}{{\selectfont\char223}}1
      {і}{{\selectfont\char105}}1
      {ї}{{\selectfont\char168}}1
      {є}{{\selectfont\char185}}1
      {ґ}{{\selectfont\char160}}1
      {І}{{\selectfont\char73}}1
      {Ї}{{\selectfont\char136}}1
      {Є}{{\selectfont\char153}}1
      {Ґ}{{\selectfont\char128}}1
      {\{}{{{\color{bracketscolor}\{}}}1
      {\}}{{{\color{bracketscolor}\}}}}1
    }
% Code caption:
%\usepackage{caption}
%    \DeclareCaptionFont{white}{\color{сaptiontext}}
%    \DeclareCaptionFormat{listing}{\parbox{\linewidth}{\colorbox{сaptionbk}{\parbox{\linewidth}{#1#2#3}}    \vskip-4pt}}
%    \captionsetup[lstlisting]{format=listing,labelfont=white,textfont=white}
\renewcommand{\lstlistingname}{Код}


\title{\normalfont\spacedallcaps{Название статьи}}
\author{\spacedlowsmallcaps{JohnSmith* \& YourName\textsuperscript{1}}}
\date{}