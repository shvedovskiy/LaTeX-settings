\documentclass[t,pdf,hyperref={unicode}]{beamer} % [t]|[c]|[b] -- вертикальное выравнивание на слайдах 
%\documentclass[handout]{beamer} % Раздаточный материал
%\documentclass[aspectratio=169]{beamer} % Соотношение сторон

\usetheme{Goettingen}
\usecolortheme{beaver}

\usepackage{cmap}			
\usepackage{mathtext} 			
\usepackage[T2A]{fontenc}		
\usepackage[utf8]{inputenc}			
\usepackage[english,russian]{babel}

\newtheorem{rtheorem}{Теорема}
\newtheorem{rproof}{Доказательство}
\newtheorem{rexample}{Пример}

\usepackage{amsmath,amsfonts,amssymb,amsthm,mathtools}
\usepackage{graphicx}
	\graphicspath{{images/}} 
	\setlength\fboxsep{3pt}
	\setlength\fboxrule{1pt}
	
\usepackage{wrapfig}
\usepackage{array,tabularx,tabulary}
\usepackage{longtable}  
\usepackage{multirow}
\usepackage{multicol} 
\usepackage{lastpage} 
\usepackage{soul} 
\usepackage{csquotes} 
\usepackage{verbatim}
\usepackage{tikz} 
\usepackage{pgfplots}
\usepackage{pgfplotstable}

% Для заполнения
\title{}
\subtitle{}
\author{}
\date{}
\institute[University]{}

\begin{document}

\frame[plain]{\titlepage}

\section{1}
\subsection{1.1}

\begin{frame}[c]
	\frametitle{\insertsection} 
	\framesubtitle{\insertsubsection}
	\begin{block}{Block Name}
		\begin{itemize}
			\item One\pause
			\item Two
		\end{itemize}
	\end{block}
	\includegraphics{File}
	\end{frame}

\subsection{1.2}
 
\begin{frame}[fragile]
	\frametitle{\insertsection} 
	\framesubtitle{\insertsubsection}
	\alert<1>{\alert<5>{\texttt{<?xml version="1.0"\ \!encoding="UTF-16"?>}}\\
	\alert<6>{\texttt{<!DOCTYPE quiz>}}\\
	\alert<7>{\texttt{<?xml-stylesheet href="stylesheet.css"?>}}\\}
	\alert<2>{\texttt{<quiz>}\\
	\alert<3>{\texttt{\quad<qanda seq="1"\!\,>}\\
	\texttt{\quad\quad<question>}\\
	\texttt{\quad\quad\quad Who was the 42-d president of the USA?}\\
	\texttt{\quad\quad</question>}\\
	\texttt{\quad\quad<answer>}\\
	\texttt{\quad\quad\quad William Jefferson Clinton}\\
	\texttt{\quad\quad</answer>}\\
	\texttt{\quad</qanda>}}\\
	\alert<4>{\texttt{\quad<!-\,\!- Note: We need to add more questions. -\,\!->}}\\
	\alert<8>{\texttt{<!\ [CDATA[if(a<b \&\& b<c) \{...\}\ ]]>}}\\
	\texttt{</quiz>}}	
\end{frame}

\begin{frame}
  \uncover<4->{Эта строчка появляется не сразу, но занимает место.}
  \only<5->{Эта строчка появляется не сразу и не занимает места.}
  \begin{enumerate}
    \item<1-5> Сначала появляются первый и последний пункт (первый потом исчезнет).
    \item<2-> Потом второй
    \item<3-> И наконец третий
    \item<1-> Последний пункт появляется вместе с первым
    \item<6-> В самом конце первый пункт исчезает, зато появляется картинка: \insertlogo.
  \end{enumerate}
  \alt<4>{Это \alert{четвертый} слайд}{Это не четвертый слайд.}
  \temporal<3-4>{Слайды 1, 2}{Слайды 3, 4}{Сайды 5, 6, 7, ...}
\end{frame}

\begin{frame}{Пример из руководства}
  \textbf{This line is bold on all three slides.}
  \textbf<2>{This line is bold only on the second slide.}
  \textbf<3>{This line is bold only on the third slide.}
  \textbf<3,4>{Эта строчка полужирная на 3-м и 4-м слайде.}
	\color<3-4>[RGB]{255,0,0} Этот текст красный только на сладах 3-4.
\end{frame}

\begin{frame}
  \begin{rtheorem}
    Формулировка теоремы.    
  \end{rtheorem}
	\begin{rproof}
		Текст доказательства.
  \end{rproof}
	\begin{rexample}
		Текст примера.
	\end{rexample}
\end{frame}

\end{document}
